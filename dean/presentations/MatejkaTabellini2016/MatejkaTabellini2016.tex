%% LyX 2.2.3 created this file.  For more info, see http://www.lyx.org/.
%% Do not edit unless you really know what you are doing.
\documentclass[english]{beamer}
\usepackage[utf8x]{inputenc}
\setcounter{secnumdepth}{3}
\setcounter{tocdepth}{3}
\usepackage{amstext}
\usepackage{esint}
\usecolortheme{dove}
%\usetheme{Ilmenau}
%\usefonttheme{structurebold}

\AtBeginSection[]
{
  \begin{frame}<beamer>
    \frametitle{Outline}
    \tableofcontents[currentsection]
  \end{frame}
}

\usepackage[english]{babel}
\usepackage{dsfont}
\newtheorem{prop}{Proposition}
\title[]{\large Electoral Competition with 
  Rationally Inattentive Voters}
\subtitle{Mat\v{e}jka and Tabellini}
\author{Miguel Acosta}
\institute{Columbia University \& USA }
\date{October 2017}



\begin{document}
\begin{frame}
\titlepage 
\end{frame}

\section{Introduction}
%%----------------------------------------------------------------------------%%
%%----------------------------------------------------------------------------%%
%%----------------------------------------------------------------------------%%
\begin{frame}{Questions} 
Voters are uninformed, oftentimes predictably so. 
\begin{itemize}
\item{\color{red}\bf How does selective ignorance of voters interact with 
  policy formation by politicians?}
\item How can the observed patterns of what voters know be 
  explained?
\item How does the knowledge of voters depend on the political 
  process? 
\item How do the patterns in voters' information influence 
  policy choices by elected representatives?
\end{itemize}
\end{frame}

%%----------------------------------------------------------------------------%%
%%----------------------------------------------------------------------------%%
%%----------------------------------------------------------------------------%%
\begin{frame}{Preview of Results}
Using a model of rationally-inattentive voters (facing costly 
information acquisition) and politicians that care \emph{only} about 
winning, the authors draw numerous conclusions.
\begin{itemize}
\item Distortions in policies arise because, in equilibrium, candidates 
  maximize ``perceived social welfare,'' 
  (which relates to how much attention is paid to various policies), rather
  than actual social welfare.
\item More attentive voters are more influential, because they are more 
  responsive to policy changes. Candidates therefore have higher incentives
  to appeal to these groups. 
\item Voters are more attentive to issues that (1) are less costly to learn
  about (2) that are more uncertain or that (3) would provide relatively
  large increases in marginal utility. 
\item $\Rightarrow$ Small groups, extreme preferences, divisive issues 
  receive more attention. Public goods don't. Efficient reforms are more
  likely in recessions and the poor are politically empowered by 
  welfare reforms.
\end{itemize}
\end{frame}

\section{Model}
%%----------------------------------------------------------------------------%%
%%----------------------------------------------------------------------------%%
%%----------------------------------------------------------------------------%%
\begin{frame}{Candidates}
Two candidates, indexed by $C$, maximize the probability of winning the election. 
\begin{itemize}
\item Choose an $M$ dimensional policy vector $\hat q_C$. 
\item The actual policy, $q_c$ is implemented 
  imperfectly:  $q_C = \hat q_C + e_C$. 
\end{itemize}
\end{frame}
%%----------------------------------------------------------------------------%%
%%----------------------------------------------------------------------------%%
%%----------------------------------------------------------------------------%%
\begin{frame}{Voters}
$N$ voters receive ($2\times M$) {\color{red}iid} signals about the policy vectors. 
\begin{itemize}
\item Choose how much attention to pay to the signals by choosing the noise of
  each signal.\footnote{Choose the $2M$ elements of 
    $\xi^G_C\equiv \frac{\sigma_C}{\sigma^2_C+\gamma_C^G}$\\}
\item Cost of attention is a weighted sum (over policies) of the relative 
  reduction of uncertainty
  that the signal provides (difference in entropy). 
\item Chooses candidate $A$ if 
\[\mathbb{E}[U^G(q_A)\mid s_A^{v,G}] - \mathbb{E}[U^G(q_A)\mid s_A^{v,G}]
\geq x^v\] where $x^v$ is a preference shock favoring candidate $B$, 
consisting of a idiosyncratic and common component. 
\item Solves $\max_{\text{info}}\mathbb{E}\left[\max_{\text{candidate}}\mathbb{E}E\left[U^G(q_C)\mid s_C^{v,G}\right]\right] -\text{info cost}$
\end{itemize}


\end{frame}
%%----------------------------------------------------------------------------%%
%%----------------------------------------------------------------------------%%
%%----------------------------------------------------------------------------%%
\begin{frame}{Timing}
  \begin{enumerate}
  \item Voters form priors about platforms and choose attention 
  \item Candidates set target policy. 
  \item Actual policies  realize. 
  \item Voters observe noisy signals of actual platforms
  \item Preference shock realizes and elections are held. 
  \end{enumerate}
\end{frame}
%%----------------------------------------------------------------------------%%
%%----------------------------------------------------------------------------%%
%%----------------------------------------------------------------------------%%
\begin{frame}{Equilibrium}
\textbf{Information and policies such that everyone maximizes utility.}

\begin{itemize}
 \item Distortions in policies arise because, in equilibrium, candidates 
  maximize ``perceived social welfare,'' 
  (which relates to how much attention is paid to various policies), rather
  than actual social welfare.
\item Voters are more attentive to issues that (1) are less costly to learn
  about (2) that are more uncertain or that (3) would provide relatively
  large increases in marginal utility. 
\end{itemize}
\end{frame}
%%----------------------------------------------------------------------------%%
%%----------------------------------------------------------------------------%%
%%----------------------------------------------------------------------------%%
\begin{frame}{Critiques}
  \begin{itemize}
  \item Signals received by voters are independent. 
  \item Voters vote as if they are pivotal. 
    \begin{itemize}
    \item Get utility from choosing the right person... ``sincere attention'' and 
      ``voting for the right candidate''
    \item Has nothing to do with who wins.
    \item ``Weight'' in information cost captures the cost of attention relative
      to the psychological benefit of voting for the right candidate.
    \end{itemize}
  \item Idiosyncratic shock being drawn late implies all voters of same group
    choose same attention strategies. 
  \end{itemize}

\end{frame}

\section{Applications \& Conclusion}
%%----------------------------------------------------------------------------%%
%%----------------------------------------------------------------------------%%
%%----------------------------------------------------------------------------%%
\begin{frame}{Applications}
  \begin{itemize}
  \item \textbf{Single policy dimension.} Rational inattention amplifies the 
    effect of preference utility and dampens the effect of group size on 
    outcomes.
    \item \textbf{Multi-dimensional policy.} Voters pay more attention to higher
      stakes. Common-ground policies receive less attention
      (expect it to prevail). 
    \item \textbf{Policy affects the cost of attention.} Welfare policies 
      increase make attention less costly for the poor $\Rightarrow$ poor
      pay more attention $\Rightarrow$ more welfare policies. 
  \end{itemize}
\end{frame}
%%----------------------------------------------------------------------------%%
%%----------------------------------------------------------------------------%%
%%----------------------------------------------------------------------------%%
\begin{frame}{Conclusion}
  \begin{itemize}
  \item Useful model which can be specialized to study several aspects of 
    elections. 
  \item The rational inattention yields several previous results, from first
    principles. 
  \end{itemize}
\end{frame}


\end{document}

%%% Local Variables:
%%% mode: latex
%%% TeX-master: t
%%% End:
